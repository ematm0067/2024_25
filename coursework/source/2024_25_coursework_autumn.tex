\documentclass[12pt]{article}
\usepackage{amsfonts, epsfig}
\usepackage{amsmath}
\usepackage{graphicx}
\usepackage{fancyhdr}
\pagestyle{fancy}
\lfoot{\texttt{emamtm0067.github.io}}
\lhead{Introduction to AI - Coursework}
\rhead{\thepage}
\cfoot{}
\begin{document}

\section*{Coursework }

\subsection*{Question 1 (25 pts)}

The idea for this question is to cluster languages using linguistic
features. For the question, you will need data from the World Atlas of
Language Structures (WALS), available at\\
\texttt{https://wals.info/}\\
Download the dataset, select
approximately 20 linguistic features, and create a matrix of languages
against these features [m1].

Once you have selected and prepared your features [m2], you should
define a similarity or distance metric between pairs of
languages. Specifically, I suggest using a simple metric based on the
count of identical feature values: the similarity between two
languages is the number of features on which they share the same value
[m3].

Use this similarity measure to perform unsupervised learning—employing
any clustering algorithm you choose—and interpret your clustering
results. Explain clearly the patterns or linguistic groups you
identify [m4].

Throughout the assignment, there are a series of marking waypoints
indicated with numbers in square brackets ([m1] to [m4]). These will
guide you on what needs to be included in your submission and how your
work will be assessed.

The distance between two languages may be greater than distance you get if
you go by way of a third language: for example, and this example is a
completely made up, image
\begin{align} 
  d(\text{Bristolian},\text{Bathish})&=34.4\\
  d(\text{Bristolian,\text{Saltfordy})&=2.8\\
  d(\text{Saltfordy},\text{Bathish})&=4.6
\end{align}
Now it is quicker to get from Bristolian to Bathish if you go by way of Saltfordy and it might be more correct to regard
\begin{equation}
  d(\text{Bristolian},\text{Bathish})=2.8+4.6=7.4
\end{equation}
Use Dijkstra's algorithm to find the shortest distance between an pair
of languages in $L$ [m5] and repeat the unsupervised learning. Is it any
different? [m6]


The mark waypoints [m1]-[m6] are there to give an idea of the marking
scheme but is intended to be a rough guide, not a rigid scheme. At
[m1] four marks will have been allocated, to get a particularly good
mark here make sure the data has been cleaned. At [m2] an additional
two marks have been allocated. At [m3] another four marks have been
allocated, be careful to explain how you are calculating the distance,
why you made that choice and what you see, for example, by noting
which Languages are closest and which furthest apart. Six more marks are
allocated by [m4], for a very good mark discuss the choice of
algorithm and any meta-parameters, use graphs to compare different
choices. Six more marks are allocated by [m5]; obviously the challenge
here is implementing the algorithm. Explain clearly what you have
done. Finally there are three more marks to be allocated by [m6].

\subsection*{Question 3 (15 marks)}

Generate a data set in two dimensions with a division boundary of the
form $y=ax^2+x$, so points one side of this boundary belong to class A
and to the other, class B. Investigate how well logistic regression
works for these data as $a$ is varied. What about a small neural
network? How are these approaches affected if the number of points is
varied, or the balance between class A and class B in the number of
points? How does changing the size of the network change the performance.


\subsection*{Question 3 (10 marks)}

Are large language models likely to make society more or less fair?
How can we effect that outcome? Discuss this in an essay of about one
page.

\begin{itemize}
\item Your essay should stay focused on that topic throughout. 
\item If you argue for or against a position, ensure your points directly support your stance. If you're presenting both sides, weigh the pros and cons.
\item Start with a clear introduction that states your main argument or the scope of your discussion.
\item Organise your points logically in the body of your essay. Each paragraph should advance your argument or analysis.
\item Conclude with a summary of your argument or final reflection on the topic.
\item Reference key studies or theories that support your argument. Provide brief explanations of why the cited studies are important for your argument.
\item  Don't just summarise; engage critically with the literature.
\item For higher marks, address counterarguments or limitations in the approaches you're discussing.
\item Aim for clear and precise writing. Avoid jargon unless necessary and explain technical terms.
\item Ensure that your essay is free of grammatical and spelling errors, and that your ideas flow smoothly from one to the next.
\end{itemize}

You can use AI in helping develop your thoughts, or for finding and
fixing errors, but I expect clarity, originality and incisive thought
with strong, clearly held views while avoiding platitudes and weasel
statements. None of this will be present in an essay which has, from
the start, been taken from an LLM.


\subsection*{Report}

Your report should be no longer than seven pages, excluding any
references. Use an 11 or 12pt font and use a standard page layout, do
not expand the page just to make it fit more text!

Your report must be submitted in pdf and should be prepared in LaTeX;
overleaf is a good approach, but not required as long as LaTeX has
been used\footnote{R-markdown and some other notebook-based
environments typeset using LaTeX, this is acceptable}. As always when
using LaTeX, give yourself over to defaults, our expectation of what a
document should look like has been conditioned on LaTeX, so it is best
not to try to override the look of the document.

Avoid code snippets in the report unless that feels like the best way
to illustrate some subtle aspect of an algorithm; do always though
consider a mathematical description if possible. You will be asked to
submit code and it may be tested to make sure it works and matches
your report. It will not, however, be marked in and of itself.

\subsection*{Submission}

The deadline for report and code: 13h00 (GMT+1) on XXX 2025-MM-DD, there
will be a submission point on Blackboard under the ``assessment,
submission and feedback'' link. Please upload the following two files:
\begin{enumerate}
\item Your report as a PDF with filename <student\_number>.pdf, where
  ``<student\_number'' is replaced by your student number, not your
  username. Upload this to the submission point ``Introduction to AI
  Coursework (Turnitin)''.
\item Your code inside a single zip file with filename
  <student\_number>.zip. Inside the zip file there should be a single
  folder containing your code, with your student number as the folder
  name. Please remove datasets and other large files to minimise the
  upload size - we only need the code itself. Upload this file to the
  submission point ``Code for Introduction to AI Coursework''.
\end{enumerate}
  
We may review your Python code by eye but your marks will be based on
the contents of your report, with the code used to check how you
carried out the experiments described in your report. We will not give
marks for the coding style, comments, or organisation of the
code. Code written in Julia or R is also acceptable as is the use of a
standard notebook format. If you are particularly keen on another
programming language let me know and I will consider this.

Please do not include your name in the report text itself: to ensure
fairness, we mark the reports anonymously.

Avoiding Academic Offences: Please re-read the university's plagiarism
rules to make sure you do not break any rules. Academic offences
include submission of work that is not your own, falsification of data
/ evidence or the use of materials without appropriate
referencing. Note that sharing your report with others is also not
allowed. These offences are all taken very seriously by the University
and we have very little leeway within the framework the University has
set out. Do not copy text directly from your sources - always rewrite
in your own words and provide a citation. Work independently -- do not
share your code or reports with others; you can, of course, discuss
your work with your classmates, but do not share text or code.

Suspected offences will be dealt with in accordance with the
University's policies and procedures. If an academic offence is
suspected in your work, you will be asked to attend an interview with
senior members of the school, where you will be given the opportunity
to defend your work. The plagiarism panel can apply a range of
penalties, depending on the severity of the offence. These include a
requirement to resubmit work, capping of grades and the award of no
mark for an element of assessment. Again, we are not in a position to
be lenient here, the academic offences procedure is not one we control.

\subsection*{Extensions and Exceptional Circumstances}

If the completion of your assignment has been significantly disrupted
by serious health conditions or personal problems, or other serious
issues, you can apply for consideration in accordance with the normal
university policy and processes. Students should refer to the guidance
and complete the application forms as soon as possible when the
problem occurs. Please see the guidance below and discuss with your
personal tutor for more advice:
\begin{itemize}
\item \texttt{www.bristol.ac.uk/students/support/academic-advice/\\
  assessment-support/request-a-coursework-extension/}
\item \texttt{www.bristol.ac.uk/students/support/academic-advice/\\assessment-support/exceptional-circumstances/}
    \end{itemize}



\end{document}
